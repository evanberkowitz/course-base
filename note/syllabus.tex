%%%
%%%     Incorporate Repository Information
%%%

\providecommand{\repositoryInformationSetup}{} % Fallback definition if compiled with `make FINAL=1`.
\repositoryInformationSetup

%%%%
%%%%    Document preparation
%%%%

\begin{document}

\title{Syllabus for \class\ --- \classTitle}
\contact
\date{\semester}

\begin{abstract}
	Organizational details, rubric.
\end{abstract}

\maketitle

\section*{GENERAL INFORMATION}

\classTitle, or \class, is\ldots for \ldots
\\
\\
\begin{tabular}{p{0.2\textwidth}p{0.8\textwidth}}%rl}
	Title			&	\classTitle
	\\
	Course Number	&	\class
	\\
	\\
	Instructor		&	Dr. Evan Berkowitz	\\
	\\
					&	e-mail: \href{mailto:}{} \\
					&	Office: 
	\\
	Day and time	&	\\	
	Location		&	\\
	\\
	Office hours	&	\\
					&	Or, by appointment. \\
					&	I'm pretty flexible if you need to talk one-on-one. \\
	\\

	Teaching Assistant
					&	name \\
					&	email \\
	Office Hours	&	\\
	\\

	Text			&	author \\
					&	title with link \\
					&	publisher \\
					&	ISBN \\
	\\
	Prerequisites	&	
\end{tabular}


\clearpage
\subsection*{Academic Integrity and collaboration}

 - Link to student handbook / conduct code / honor thing.
The code explicitly names cheating, fabrication, facilitating others' academic dishonesty, plagiarism, and self-plagiarism as some categories of conduct that can earn you an XF.
Please familiarize yourself with the student conduct codes so that you know generally what is expected of students.

The University adopted an honor pledge
\begin{quote}
	``Here goes the honor pledge''
\end{quote}
For practical reasons, I will not ask you to sign it on each assignment.
However, part of Homework 00 will be to sign an all-homework-encompassing equivalent statement,
\begin{quote}
	``Equivalent''
\end{quote}

That being said, physics is a collaborative activity.
Published journal articles almost always have multiple authors who worked together to solve a problem, and most papers have an `Acknowledgements' section that (in addition to noting funding sources, for example) thanks people who, through conversation, helped clarify the authors' understanding and results.
Similarly, you may work together to understand homework assignments, as long as you write who your collaborators are.
However, in contrast to a journal publication, you cannot turn in joint solutions---the solutions should reflect your own understanding.

In practice, what that means is that if you work out a problem together, you should write up your solutions without looking at anybody else's solution.
If you share space and work out a problem at a board, you might take notes on the solution as you hash it out, separate, and prepare your written solutions separately.

\section{Organizational Details}

 - Where, when, remote details, pre-recorded lectures, attendance.

\subsection*{Zoom}

How to connect, a user guide, camera-on policy, the chat.

If we're zooming, I will often be sharing my screen.  Zoom usually defaults to full-screen when someone shares their screen.  I recommend using Side-by-side Mode, which can be found in your Zoom preferences under `Share Screen', and Gallery View, so you can see what I am sharing, me, and your classmates.  Side-by-side mode also lets you dock the Chat and Participants to the meeting window, which only works if you're NOT in full screen.

\section{Material}

\subsection*{Book}

Book, title, publisher, link to book store, link to amazon, option for a digital copy, trial copy?.
For my own use I prefer a real copy because I find most publishers' textbook apps bordering on unusable, but I understand that a real copy is expensive, while the E-Book rental is much cheaper.
Believe you me I think academic publishing is insane, and I'm sorry.
Anyway, it's up to you.

Which portion of the book will we look at?
Throughout the term I may find supplementary digital resources from across the internet that I will link to from time to time.

Other supplemental books and links.

\section*{Assumed Background}

The prerequisites for this course are...
This material is assumed.
However, if you feel that something is egregiously missing from your preparation, please indicate it on Homework 00.

\section*{Goals for the semester}

\begin{itemize}

	\item A long list of questions that hopefully students will be able to answer.
	
	\item One for each `unit' of the course.

\end{itemize}


\clearpage
\section*{Approximate Schedule}

This schedule is approximate; the actual schedule will depend on precisely how much progress we make in each class.  However, the exam dates are fixed.

\begin{tabular}{clp{0.75\textwidth}}
	Week	&	Dates				&	Material \\\hline
	1		&&	\\
	2		&&	\\
			&&	{\bf Exam}
\end{tabular}
\\
\\
Any additional business regarding the schedule.

\clearpage
\section*{Homework Assignments}

Homework will usually be assigned on DAY, due the following DAY.
I will not assign homework that is due on exam days.
That leaves us with N homework assignments.

Sometimes for homework I will ask you to fill in some details from lecture or prove a result that I used---I will put these as exercises on the homework; you don't have to catch them when I mention them in class.
Some problems will be drawn from the book, and some not.
{\bf DO NOT LOOK UP TEXTBOOK SOLUTIONS.}

Please look at the assignments more than one day before they are due.
I hope to ask well-formulated questions, but sometimes things need clarification, and the more time I have to repair an error or clarify an issue the more time you have to succeed.

You are expected to complete all of the assigned questions.
For practical reasons the grader might not grade them all.
However, I try not to just assign ``busy work''.
I may point out good practice problems on most homework assignments which you can do at your option if you feel you need rote practice, but they are optional.

Late homework is accepted without penalty only in exceptional circumstances (an illness, a religious observance, or some other compelling reason).
Obviously not all valid excuses are foreseeable, but please let me know ahead of time if you can; this includes religious observances.
You may still turn in late homework; it will be marked but counted with a penalty.

\section*{Exams}

How many?  When?  Are they on the same day?  By zoom?  Regular room?

There will also be a final exam.  When / contingency on date from the registrar?

I intend to make the final exam simply cover the last portion of the course.  However, many of the techniques you will learn are cumulative, so doing well on the last exam will nevertheless require mastery of the earlier material.

If you require special accommodation, please reach out as soon as possible so that we may plan appropriately.

\section*{Final Grade}

Your final grade will be calculated as follows.

\begin{center}
\begin{tabular}{cl}
	20\%	&	Ten best homework scores. \\
	25\%	&	1st mid-term exam. \\
	25\%	&	2nd mid-term exam. \\
	30\%	&	3rd and final exam.
\end{tabular}
\end{center}

How that cumulative number corresponds to a letter grade will be decided later.  For example if one of the exams is much too hard and everybody does poorly, that is my fault and not yours, and shouldn't be reflected in your grade.

\section*{General Comments}

Physics is a mathematical subject, and it is definitely a cumulative subject.
Knowledge you acquire this semester will be built on your prior knowledge, and what you hope to learn in the future will likely depend on your familiarity with, if not your mastery of, what you learn this semester.
Indeed, even what you learn later in the course will build on what you learned earlier in the semester.

Therefore, {\bf if you fall behind, find yourself in trouble, don't understand something, or feel that something was unclear, be proactive!}
Send me an e-mail, come to office hours, pop by my office when I'm around, or schedule a one-on-one meeting with me.
For your own sake, do not wait until just before the exam!

\bibliography{}

\end{document}

\usepackage{enumitem}
\usepackage{amsmath,amssymb}
\usepackage{siunitx}
\usepackage{bm}
\usepackage{graphicx}
\usepackage[dvipsnames]{xcolor}
\usepackage{slashed}
\usepackage{xspace}
\usepackage{bbm}
\usepackage{ifthen}
\usepackage{hyperref}
\hypersetup{
    colorlinks=true,
    linkcolor=blue,
    filecolor=blue,
    urlcolor=blue,
    }

\usepackage{tikz}
\usetikzlibrary{decorations.markings}
\usetikzlibrary{patterns}
\usepackage{pgfplots}

%%%%
%%%%    Project Specifics
%%%%

\newcommand{\repoURL}{https://github.com/evanberkowitz/course-base}
\newcommand{\myEmail}{email address}
\newcommand{\contact}{\author{Dr.~Evan Berkowitz \\
\href{mailto:\myEmail}{\myEmail}}}
\newcommand{\class}{CLASS\xspace}
\newcommand{\classTitle}{LONG TITLE\xspace}
\newcommand{\semester}{Semester, Year\xspace}

%%%%
%%%%    Referring to Parts of the Document
%%%%

\newcommand{\secref}[1]{Sec.~\ref{sec:#1}}
\newcommand{\Secref}[1]{Section~\ref{sec:#1}}
\newcommand{\appref}[1]{App.~\ref{sec:#1}}
\newcommand{\Appref}[1]{Appendix~\ref{sec:#1}}
\newcommand{\tabref}[1]{Tab.~\ref{tab:#1}\xspace}
\newcommand{\Tabref}[1]{Table~\ref{tab:#1}\xspace}
\newcommand{\figref}[1]{Fig.~\ref{fig:#1}\xspace}
\newcommand{\Figref}[1]{Figure~\ref{fig:#1}\xspace}
\renewcommand{\eqref}[1]{(\ref{eq:#1})\xspace}
\newcommand{\Eqref}[1]{Equation~\ref{eq:#1}\xspace}
\newcommand{\eq}[1]{#1~\eqref{#1}}
\def\Ref#1{Ref.~\cite{#1}} % Use \def because some LaTeX installations define \Ref, others do not.
\newcommand{\Reference}[1]{Reference~\cite{#1}}
\newcommand{\Refs}[1]{Refs.~\cite{#1}}
\newcommand{\References}[1]{References~\cite{#1}}

%%%%
%%%%    Referring to Parts of the GitHub Repository
%%%%

\newcommand{\issue}[1]{\href{\repoURL/issues/#1}{Issue #1}}
\newcommand{\pullrequest}[1]{\href{\repoURL/pulls/#1}{Pull Request #1}}

%%%%
%%%%    Referring to Other Documents
%%%%

\providecommand{\doi}[1]{\href{http://doi.org/#1}{[#1]}}
\newcommand{\arxiv}[1]{\href{http://www.arxiv.org/abs/#1}{arXiv:#1}}

%%%%
%%%%    Mathematical Symbols
%%%%

\newcommand{\goesto}{\ensuremath{\rightarrow}}
\newcommand{\infinity}{\infty}
\newcommand{\Integers}{\mathbb{Z}\xspace}
\newcommand{\integers}{\Integers}
\newcommand{\one}{\ensuremath{\mathbbm{1}}}
\newcommand{\order}[1]{\ensuremath{\mathcal{O}\left(#1\right)}\xspace}
\newcommand{\Rationals}{\mathbb{Q}\xspace}
\newcommand{\Reals}{\mathbb{R}\xspace}
\newcommand{\Complexes}{\mathbb{C}\xspace}
\newcommand{\union}{\ensuremath{\cup}}
\DeclareMathOperator{\erf}{erf}
\newcommand{\laplace}[1]{\ensuremath{\mathcal{L}\left\{#1\right\}}\xspace}
\newcommand{\inverselaplace}[1]{\ensuremath{\mathcal{L}\inverse\left\{#1\right\}}\xspace}
\providecommand{\mod}[1]{\ensuremath{\ \left(\text{mod }#1\right)}}

\DeclareMathOperator{\odd}{odd}
\DeclareMathOperator{\even}{even}
\DeclareMathOperator{\sinc}{sinc}
\DeclareMathOperator{\real}{Re}
\DeclareMathOperator{\imag}{Im}

%%%%
%%%% Vectors + Geometry
%%%%

\newcommand{\degrees}{\ensuremath{{}^{\circ}}}
\newcommand{\ihat}{\,\hat{\textbf{\i}}}
\newcommand{\jhat}{\,\hat{\textbf{\j}}}
\newcommand{\khat}{\,\hat{\textbf{k}}}
\newcommand{\cross}{\times}

%%%%
%%%%    Hyperbolic Trig
%%%%

% Most are already available if you \usepackage{amsmath}.
% However, those below are missing

\DeclareMathOperator{\sech}{sech}
\DeclareMathOperator{\csch}{csch}
\DeclareMathOperator{\arccosh}{arccosh}
\DeclareMathOperator{\arcsinh}{arcsinh}
\DeclareMathOperator{\arctanh}{arctanh}
\DeclareMathOperator{\arcsech}{arcsech}
\DeclareMathOperator{\arccsch}{arccsch}
\DeclareMathOperator{\arccoth}{arccoth}

% Additionally, there are some missing trig functions:

\DeclareMathOperator{\arcsec}{arcsec}
\DeclareMathOperator{\arccot}{arccot}
\DeclareMathOperator{\arccsc}{arccsc}

%%%%
%%%%    Fractions
%%%%

\newcommand{\oneover}[1]{\ensuremath{\frac{1}{#1}}}                             %   1/[argument]
\newcommand{\inverse}{\ensuremath{^{-1}}}                                       %   argument^-1
\providecommand{\half}{\ensuremath{\frac{1}{2}} }                               %   1/2
\renewcommand{\half}{\ensuremath{\frac{1}{2}} }                                 %   1/2
\newcommand{\quarter}{\ensuremath{\frac{1}{4}} }                                %   1/4

%%%%
%%%%    Derivatives
%%%%

% d^n f / dx^n
\newcommand{\dd}[3][1]{
    \ifthenelse { \equal {#1} {1} }
                {\ensuremath{\frac{d#2}{d#3}}}
                {\ensuremath{\frac{d^{#1}#2}{d#3^{#1}}}}
    }

% partial^n f / partial x^n
\newcommand{\pp}[3][1]{
    \ifthenelse { \equal {#1} {1} }
                {\ensuremath{\frac{\partial#2}{\partial#3}}}
                {\ensuremath{\frac{\partial^{#1}#2}{\partial#3^{#1}}}}
    }

% partial f / partial x partial y
\newcommand{\ppp}[3]{\ensuremath{\frac{\partial^2#1}{\partial#2\,\partial#3}}}

% Vector Derivatives 
\newcommand{\grad}{\ensuremath{\nabla}\xspace}
\newcommand{\laplacian}{\ensuremath{\grad^2}\xspace}

% Integration d
\providecommand{\id}{}
\renewcommand{\id}[1]{\ensuremath{\; \mathrm{d}#1}}

%%%%
%%%%    Mathematical Delimiters
%%%%
\newcommand{\abs}[1]{\ensuremath{\left| #1 \right|}\xspace}
\newcommand{\magnitude}{\abs}
\newcommand{\average}[1]{\ensuremath{\left\langle #1 \right\rangle}\xspace}

% Quantum mechanics bra-ket notation:
\newcommand{\ket}[1]{\ensuremath{\left|\;#1\;\right\rangle}}
\newcommand{\bra}[1]{\ensuremath{\left\langle\;#1\;\right|}}
\newcommand{\bracket}[2]{\ensuremath{\left\langle\;#1\;\middle|\;#2\;\right\rangle}}
\let\braket\bracket
\newcommand{\operator}[3]{\ensuremath{\left|\;#1\;\middle\rangle\; #2\; \middle\langle\;#3\;\right|}}

%%%%
%%%%    Matrices
%%%%

\newcommand{\identity}{\ensuremath{\mathds{1}}}
\newcommand{\diag}[1]{\ensuremath{\text{diag}\left(#1\right)}}
\newcommand{\tr}[1]{\ensuremath{\text{tr}\left[#1\right]}}
\newcommand{\transpose}{\ensuremath{{}^{\top}}}
\newcommand{\adjoint}{\ensuremath{{}^{\dagger}}}

%%%%
%%%%    Physical Quantities and Constants
%%%%




%%%%
%%%%    Software
%%%%

\newcommand{\bash}{\texttt{bash}\xspace}
\newcommand{\git}{\texttt{git}\xspace}
\newcommand{\make}{\texttt{make}\xspace}
\newcommand{\mpi}{\texttt{MPI}\xspace}
\newcommand{\python}{\texttt{python}\xspace}

% Put an xspace after \LaTeX:
\let\builtinLaTeX\LaTeX
\def\LaTeX{\builtinLaTeX\xspace}
